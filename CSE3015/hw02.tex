\documentclass{homework}

\usepackage[a4paper,margin=1in]{geometry}
\usepackage{kotex}
\usepackage{environ}

\usepackage{amsmath}
\usepackage{amssymb}
\usepackage{braket}

\usepackage{graphicx}

\usepackage{tikz}

\usepackage{karnaugh-map}

\newcommand{\hwname}{이주헌}
\newcommand{\hwemail}{20191629}
\newcommand{\hwnum}{2}

\newcommand{\hwtype}{Homework}
\newcommand{\hwclass}{CSE3015}

\begin{document}

\maketitle

\question*{Let $F(A, B, C, D) = BD + B'D'$ is given.}

\begin{itemize}
\item Make given function to canonical SOP form.

The truth table for the function $F$ is:

\begin{tabular}{cccc|c|c}
$A$ & $B$ & $C$ & $D$ & $F$ & Minterm \\
\hline
T & T & T & T & T & $m_{15}$ \\
T & T & T & F & F & $m_{14}$ \\
T & T & F & T & T & $m_{13}$ \\
T & T & F & F & F & $m_{12}$ \\
T & F & T & T & F & $m_{11}$ \\
T & F & T & F & T & $m_{10}$ \\
T & F & F & T & F & $m_9$ \\
T & F & F & F & T & $m_8$ \\
F & T & T & T & T & $m_7$ \\
F & T & T & F & F & $m_6$ \\
F & T & F & T & T & $m_5$ \\
F & T & F & F & F & $m_4$ \\
F & F & T & T & F & $m_3$ \\
F & F & T & F & T & $m_2$ \\
F & F & F & T & F & $m_1$ \\
F & F & F & F & T & $m_0$ \\
\end{tabular}

From the truth table above, the 1-minterm indices are $\{0, 2, 5, 7, 8, 10, 13, 15\}$. Hence, the canonical SOP form can be written as:

$$
F_1(A, B, C, D) = \sum{(0, 2, 5, 7, 8, 10, 13, 15)}
= m_0 + m_2 + m_5 + m_7 + m_8 + m_{10} + m_{13} + m_{15}
$$

\item{With canonical SOP form, fill up the Karnaugh Map.}

\begin{center}
\begin{karnaugh-map}[4][4][1][$AB$][$CD$]
\minterms{0, 2, 5, 7, 8, 10, 13, 15}
\end{karnaugh-map}
\end{center}

\item{Using the Karnaugh Map, find the simplest(shortest) SOP form. Show each step and detail.}

The Karnaugh Map given by the previous step looks like this:

\begin{center}
\begin{karnaugh-map}[4][4][1][$AB$][$CD$]
\minterms{0, 2, 5, 7, 8, 10, 13, 15}
\end{karnaugh-map}
\end{center}

Here, it is pretty obvious that we can group the four 1's in the middle into an implicant group as follows:

\begin{center}
\begin{karnaugh-map}[4][4][1][$AB$][$CD$]
\minterms{0, 2, 5, 7, 8, 10, 13, 15}
\implicant{5}{15}
\end{karnaugh-map}
\end{center}

Considering that Karnaugh Map can ``wrap around'' in all four directions, we can connect the four corner cells into an implicant group as well:

\begin{center}
\begin{karnaugh-map}[4][4][1][$AB$][$CD$]
\minterms{0, 2, 5, 7, 8, 10, 13, 15}
\implicant{5}{15}
\implicantcorner
\end{karnaugh-map}
\end{center}

Since all two implicants are not included in any other implicants, and include minterms that are not included in any other prime implicants, we can say they all are \textbf{essential prime implicants}. 

The red implicant represents $BD$, and the green implicant represents $B'D'$. Therefore, connecting these two implicants with logical OR gives the following shortest SOP form:
$$
F(A, B, C, D) = BD + B'D'
$$
\end{itemize}

\question*{Let $G(A, B, C, D) = (A + C)(B' + C' + D')(B + C' + D)$ is given.}

\begin{itemize}
\item Make given function to canonical POS form.

The truth table for the function $G$ is:

\begin{tabular}{cccc|c|c}
$A$ & $B$ & $C$ & $D$ & $G$ & Maxterm \\
\hline
T & T & T & T & F & $M_{15}$ \\
T & T & T & F & T & $M_{14}$ \\
T & T & F & T & T & $M_{13}$ \\
T & T & F & F & T & $M_{12}$ \\

T & F & T & T & T & $M_{11}$ \\
T & F & T & F & F & $M_{10}$ \\
T & F & F & T & T & $M_9$ \\
T & F & F & F & T & $M_8$ \\

F & T & T & T & F & $M_7$ \\
F & T & T & F & T & $M_6$ \\
F & T & F & T & F & $M_5$ \\
F & T & F & F & F & $M_4$ \\

F & F & T & T & T & $M_3$ \\
F & F & T & F & F & $M_2$ \\
F & F & F & T & F & $M_1$ \\
F & F & F & F & F & $M_0$ \\
\end{tabular}

From the truth table above, the 0-maxterm indices are $\{0, 1, 2, 4, 5, 7, 10, 15\}$. Hence, the canonical POS form can be written as:

$$
G_1(A, B, C, D) = \prod{(0, 1, 2, 4, 5, 7, 10, 15)}
= M_0M_1M_2M_4M_5M_7M_{10}M_{15}
$$

\item{With canonical POS form, fill up the Karnaugh Map.}

\begin{center}
\begin{karnaugh-map}[4][4][1][$CD$][$AB$]
\maxterms{0, 1, 2, 4, 5, 7, 10, 15}
\end{karnaugh-map}
\end{center}

\item{Using the Karnaugh Map, find the simplest(shortest) SOP form. Show each step and detail.}

The Karnaugh Map given by the previous step looks like this:

\begin{center}
\begin{karnaugh-map}[4][4][1][$CD$][$AB$]
\maxterms{0, 1, 2, 4, 5, 7, 10, 15}
\end{karnaugh-map}
\end{center}

The upper-left corner has four 0-maxterms arranged like a square, so we can put them in an implicant group.

\begin{center}
\begin{karnaugh-map}[4][4][1][$CD$][$AB$]
\maxterms{0, 1, 2, 4, 5, 7, 10, 15}
\implicant{0}{5}
\end{karnaugh-map}
\end{center}

On the third column, there are two maxterms in a line. They can put into an implicant group as well.

\begin{center}
\begin{karnaugh-map}[4][4][1][$CD$][$AB$]
\maxterms{0, 1, 2, 4, 5, 7, 10, 15}
\implicant{0}{5}
\implicant{7}{15}
\end{karnaugh-map}
\end{center}

Again, considering that Karnaugh Map can ``wrap around'' in all four directions, we can put two in the last column into their own implicant group.

\begin{center}
\begin{karnaugh-map}[4][4][1][$CD$][$AB$]
\maxterms{0, 1, 2, 4, 5, 7, 10, 15}
\implicant{0}{5}
\implicant{7}{15}
\implicantedge{2}{2}{10}{10}
\end{karnaugh-map}
\end{center}

Since all two implicants are not included in any other implicants, and include maxterms that are not included in any other prime implicants, we can say they all are \textbf{essential prime implicants}. 

For the red implicant, we find terms $A'$ and $C'$. Taking their complement and connecting them with logical OR, we find the red implicant represents $A + C$. For the green implicant, we find terms $B$, $C$, and $D$. Following the same procedure, we find the green implicant represents $B' + C' + D'$. Similarly, the yellow implicant represents $B + C' + D$. Connecting all terms with logical AND, we get the shortest POS form:
$$
G(A, B, C, D) = (A + C)(B' + C' + D')(B + C' + D)
$$
\end{itemize}

\end{document}