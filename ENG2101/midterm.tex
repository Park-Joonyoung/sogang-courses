\documentclass{mla}

\firstname{Juhun}
\lastname{Lee}
\instructor{Professor Jay Jeon}
\course{Writing about English Literature}
\date{26 April 2023}

\addbibresource{midterm.bib}

\title{Harnessing the Power of ``I''}

\begin{document}
Through the first-person description of women's experiences, ``The Bloody Chamber'' by Angela Carter and ``The Husband Stitch'' by Carmen Maria Machado depict oppressive domestic relationships.
Both stories feature unnamed female protagonists who reveal why they become involved in such relationships at the opening.
The narrator in ``The Bloody Chamber'' has financial motivations to begin her relationship with the Marquis.
Meanwhile, the narrator in ``The Husband Stitch'' expresses how passionately in love she is with her soon-to-be husband.
While they have vastly different motivations behind their marriages, their experiences share a common theme of subjugation and harassment.
Throughout ``The Bloody Chamber'', the narrator allows the reader to construct a vivid mental image of the story with detailed descriptions of the immediate surroundings and her thoughts.
She rarely interjects her narration, and supplementary literary devices such as symbols are introduced only to increase the effectiveness of her delivery.
On the other hand, Machado interweaves the narrator's experiences and communal stories with metanarrative directions in ``The Husband Stitch'' to indicate the protagonist's emotional state.
Despite these differences in their use of literary devices, both short stories expose the harmful effects of patriarchal power dynamics through the use of a first-person point of view.

It is well-known that ``First-person narrators\ldots are called `ironic' or `unreliable' narrators'' \autocite[][175]{glossary}, since the narrator, who also experiences the events in the story, may introduce biases to their descriptions.
However, the narrator of ``The Bloody Chamber'' is not unreliable as she expresses her emotion accurately, especially that of powerlessness and fear.
In the story, when the narrator is forcefully stripped by the Marquis, she confesses: ``I was aghast to feel myself stirring'' \autocite[][12]{carter}.
Her feeling of being ``stirred'' is foreshadowed in the previous text where she ``[senses]\ldots a potentiality for corruption'' \autocite[][7]{carter} the day before her marriage.
This confession provides a useful insight into the narrative strategy of the short story.
Generally, a confession related to an unexpected sensation of pleasure from an unwanted and perverted action is rare in real-world social interactions, as such confession complicates human relationships.
Although the narrator is not a real person, the introduction of this rare, \textit{personal} experience hints to the readers both the extent to which the narrator shares her emotional state and the genuineness of such descriptions.
Thus, throughout the story, the first-person narration of ``The Bloody Chamber'' serves as the single source of emotional truth in the manipulative relationship the protagonist involved in.
With the pronoun ``I'', Carter sets up a groundwork to unearth the destructive nature of patriarchal power dynamics.

The use of first-person narration in ``The Bloody Chamber'' not only provides vivid images of the situation but also highlights the traumatic experiences of the narrator.
In the description of the narrator's loss of virginity: ``a dozen husbands impaled a dozen brides while the mewing gulls swung on invisible trapezes in the empty air outside'' \autocite[][15]{carter}.
Unlike other parts of the story, the narrator deliberately refers to herself as ``a dozen brides''.
This type of ``depersonalization'', a phenomenon where one becomes a detached observer of oneself, is often found in those who have experienced acute traumatic experiences \autocite{dissociation}.
The narrator, who is willing to share her honest emotional state as demonstrated above, breaks her promise to the readers and distances herself from the event.
Carter implies the mental scar to the narrator is irreversible by temporarily shifting the point of view.
The one-directional relationship is further established with the juxtaposition of her trauma with the Marquis leaving France for ``a deal, an enterprise of hazard and chance involving several millions'' \autocite[][16]{carter}.
To the Marquis, no matter how wounded his wife is, she is just an object that belongs to him, not a family member with a say in domestic affairs.
Therefore, the use of first-person narration---or rather, a deliberate and temporary departure from one---ultimately highlights the traumatic experiences of the narrator and the unidirectional power dynamics in patriarchal relationships.

On the other hand, the narrator of ``The Husband Stitch'' interlaces her recollections with ``stage directions'' to convey the subtle effect of male dominance.
One manifestation of this is the narrator's story about her son's growth. 
When her newborn son plays with her ribbon, she suggests that her son ``thinks of [the ribbon] as a part of [her], and he treats it no differently than he would an ear or a finger'' \autocite[][18]{machado}.
Later, when her son reaches five and attempts to touch it again, she ``[shakes] a can full of pennies. It crashes discordantly and he withdraws and weeps'' \autocite[][21]{machado}.
Unlike her husband's attempt to touch it, she allows her son to play with her ribbon when he is young.
However, when her son eavesdrops on the couple's conversation and attempts to touch it again, she feels a great threat and rejects him.
This scene marks the beginning of her son's transformation from a part of her body, which is allowed to touch her green ribbon, to a man, a part of the oppressing group that forces certain behavior upon the narrator.
Furthermore, the fact that her son's action is essentially provoked by her husband's constant request also represents the nature of patriarchy---it is acquired from male members of a family.
Therefore, while part of its intention is to portray the reaction of her son, the narrator's instruction to ``prepare a soda can full of pennies\ldots [and] shake it loudly in the face of the people closest to you'' \autocite[][21]{machado} ultimately mirrors the feel of ``betrayal'' by her son she experiences.
By introducing a stage direction aimed at the readers after personal recollections of events, Machado enhances the emotional impact and suggests a novel interpretation.

Machado also interweaves the narrator's voice with folklore to foreshadow the destruction caused by an oppressive relationship.
When she announces to her husband that she is pregnant, he attempts to touch her ribbon, ``grabbing [her] wrists with one hand as he touches the ribbon with the other'' \autocite[][12]{machado}.
The narrator juxtaposes her husband's forceful attempt to untie her ribbon with a communal story ``about a pioneer husband and wife killed by wolves. Neighbors found their bodies\ldots, but never located their infant daughter'' \autocite[][13]{machado}.
The textual proximity allows the reader to correlate the husband's predatory behavior and the wolves killing the pioneer couple.
However, in this context, the narrator's husband is not equated to the wolves---his behavior is.
Afterward, the narrator proceeds to reveal what happens to the then-infant daughter: ``she was said to be seen resting in the rushes along a riverbank, suckling two wolf cubs'' \autocite[][13]{machado}.
The narrator adds that the daughter must have ``felt a kind of sanctuary, peace she would have found nowhere else'' \autocite[][13]{machado}''.
It is a clear foreshadowing of the transformation of the narrator's son, who is represented as a wolf cub.
Even when caressed and raised by a human, a wolf cub is destined to have the same predatory instinct, as the narrator's son who eventually internalized the behavior of his father.
Therefore, with the juxtaposition of a traumatic event and a communal story, Machado demonstrates the infectious nature of patriarchal relationships.

First-person narration is especially effective in building immersion into the main protagonist.
Carter and Machado, in their short stories ``The Bloody Chamber'' and ``The Husband Stitch'', feature female protagonists telling stories of oppressive relationships.
Carter uncovers the nature of patriarchal power dynamics by creating an honest narrator and carefully portraying her mental state with the tone of the narration.
Meanwhile, Machado uses a different approach: her narrator is not as emotionally open to the readers as Carter's character, but she intertwines folklore and stage directions to introduce deeper insights about seemingly literal recollections.
Although these two stories use different tactics with the first-person point of view, ultimately both stories expose the harmful effects of patriarchal power dynamics.
\end{document}